\documentclass{llncs}
%\usepackage[margin=1in,a4paper]{geometry}
\usepackage[T1]{fontenc}
\usepackage{palatino}
\usepackage{xspace}
\usepackage{microtype}
\usepackage{tikz,eurosym}
\usepackage{amsmath,amssymb}
\usepackage{enumitem}
\usetikzlibrary{shapes,arrows}
\usetikzlibrary{positioning}
\usetikzlibrary{calc}

% Relate to:
% http://fc14.ifca.ai/papers/fc14_submission_124.pdf

% Terminology:
% - SEPA-transfer -- avoid 'SEPA transaction' as we use
%       'transaction' already when we talk about taxable
%        transfers of Taler coins and database 'transactions'.
% - wallet = coins at customer
% - reserve = currency entrusted to exchange waiting for withdrawal
% - deposit = SEPA to exchange
% - withdrawal = exchange to customer
% - spending = customer to merchant
% - redeeming = merchant to exchange (and then exchange SEPA to merchant)
% - refreshing = customer-exchange-customer
% - dirty coin = coin with exposed public key
% - fresh coin = coin that was refreshed or is new
% - coin signing key = exchange's online key used to (blindly) sign coin
% - message signing key = exchange's online key to sign exchange messages
% - exchange master key = exchange's key used to sign other exchange keys
% - owner = entity that knows coin private key
% - transaction = coin ownership transfer that should be taxed
% - sharing = coin copying that should not be taxed


\title{Post-quantum anonymity in Taler}

\begin{document}
\mainmatter

\author{Jeffrey Burdges}
\institute{Intria / GNUnet / Taler}


\maketitle

\begin{abstract}
David Chaum's original RSA blind sgnatures provide information theoretic
anonymity for customers' purchases.  In practice, there are many schemes
that weaken this to provide properties.  We describe a refresh protocol
for Taler that provides customers with post-quantum anonymity.
It replaces an elliptic curve Diffe-Hellman operation with a unique
hash-based encryption scheme for the proof-of-trust via key knoledge
property that Taler requires to distinguish untaxable operations from
taxable purchases. 
\end{abstract}


\section{Introduction}

David Chaum's RSA blind sgnatures \cite{} can provide financial
security for the exchange, or traditionally mint,
 assuming RSA-CTI \cite{,}. 

A typical exchange deployment must record all spent coins to prevent
double spending.  It would therefore rotate ``denomination'' signing
keys every few weeks or months to keep this database from expanding
indefinitely \cite{Taler??}.  As a consequence, our exchange has
ample time to respond to advances in cryptgraphy by increasing their
key sizes, updating wallet software with new algorithms, or
even shutting down.

In particular, there is no chance that quantum computers will emerge
and become inexpensive within the lifetime of a demonination key.
Indeed, even a quantum computer that existed only in secret posses
little threat because the risk of exposing that secret probably exceeds
the exchange's value. 

\smallskip

We cannot make the same bold pronouncement for the customers' anonymity
however.  We must additionally ask if customers' transactions can be
deanonymized in the future by the nvention of quantum computes, or
mathematical advances. 

David Chaum's original RSA blind sgnatures provide even information
theoretic anonymity for customers, giving the desired negative answer.
There are however many related schemes that add desirable properties
at the expense of customers' anonymity.  In particular, any scheme
that supports offline merchants must add a deanonymization attack
when coins are double spent \cite{B??}.  

Importantly, there are reasons why exchanges must replace coins that
do not involve actual financial transactons, like to reissue a coin
before the exchange rotates the denomination key that signed it, or
protect users' anonymity after a merchant recieves a coin, but fails
to process it or deliver good.

In Taler, coins can be partially spent by signing with the coin's key
for only a portion of the value determined by the coin's denomination
key.  This allows precise payments but taints the coin with a
transaction,  which frequently entail user data like a shipng address.  
To correct this, a customer does a second transaction with the exchange
where they sign over the partially spent coin's risidual balance
in exchange for new freshly anonymized coins.  
Taler employs this {\em refresh} or {\em melt protocol} for
both for coins tainted through partial spending or merchant failures,
as well as for coin replacement due to denomination key roration.

If this protocol were simply a second transaction, then customers
would retain information theoreticaly secure anonymity.  
In Taler however, we require that the exchange learns acurate income
information for merchants.  If we use a regular transaction, then
a customer could conspire to help the merchant hide their income
\cite[]{Taler??}.
To prevent this, the refresh protocol requires that a customer prove
that they could learn the private key of the resulting new coins.

At this point, Taler employs an elliptic curve Diffie-Hellman key
exchange between the coin's signing key and a new linking key 
\cite[??]{Taler??}.  As the public linking key is exposed,
an adversary with a quantum computer could trace any coins involved
in either partial spending operations or aborted transactions.
A refresh prompted by denomination key rotation incurs no anonymity
risks regardless.

\smallskip

We propose two variations on Taler's refresh protocol that offer
resistane to a quantum adversary.

First, we describe attaching contemporary post-quantum key exchanges,
based on either super-singular eliptic curve isogenies \cite{SIDH} or
ring learning with errors (Ring-LWE) \cite{Peikert14,NewHope}.
These provide strong post-quantum security so long as the underlying
scheme remains secure; however, these schemes youth leaves them
relatively untested.

Second, we propose a hash based scheme whose anonymity garentee needs
only the one-way assumption on our hash function.  In this scheme,
the vible security paramater is numerically far smaller than in the
key exchange systems, but covers query complexity which we believe
suffices.

We describe this hash based proof-of-encryption-to-self scheme in
parallel with the 
As is the practice with hash based signature schemes 




In this paper, we describe a post-quantum 

It replaces an elliptic curve Diffe-Hellman operation with a unique
hash-based encryption scheme for the proof-of-trust via key knoledge
property that Taler requires to distinguish untaxable operations from
taxable purchases. 

...

\smallskip

We observe that several elliptic curve blind signature schemes provide
information theoreticly secure blinding as well, but 
 Schnorr sgnatures require an extra round trip \cite{??}, and
 pairing based schemes offer no advnatages over RSA \cite{??}.

There are several schemes like Anonize \cite{} in Brave \cite{}, 
or Zcash \cite{} used in similar situations to blind signatures. 
% https://github.com/brave/ledger/blob/master/documentation/Ledger-Principles.md
In these systems, anonymity is not post-quantum due to the zero-knowledge
proofs they employ.


\section{Taler's refresh protocol}

\def\Mu{M}
\def\Eta{}
\def\newmathrm#1{\expandafter\newcommand\csname #1\endcsname{\mathrm{#1}}}
\newmathrm{CPK}
\newmathrm{CSK}
\newmathrm{LPK}
\newmathrm{LSK}
\newmathrm{KEX}


We shall describe Taler's refresh protocol in this section.
All notation defined here persists throughout the remainder of
 the article.

We let $\kappa$ denote the exchange's taxation security parameter,
meaning the highest marginal tax rate is $1/\kappa$.  Also, let 
$\theta$ denote the maximum number of coins returned by a refresh.

\smallskip

We label place holders $\eta$, $\lambda$, $\Lambda$, $\mu$, and $\Mu$
for key material involved in post-quantum operations.  
We view $\Lambda$ and $\Mu$ as public keys with respective
 private keys $\lambda$ and $\mu$, and
$\eta$ as the symetric key resulting from the key exchange between them.

We need effeciently computable functions
  $\CPK$, $\CSK$, $\LPK$, $\LSK$, $\KEX_2$ and $\KEX_3$  such that 
\begin{itemize}
\item  $\mu = \CSK(s)$ for a random bitstring $s$,
       $\Mu = \CPK(\mu)$,
\item  $\lambda = \LSK(t,\mu)$ for a bitstring $t$, 
       $\Lambda = \LPK(\lambda)$, and
\item $\eta = \KEX_2(\lambda,\Mu) = \KEX_3(\Lambda,\mu)$.
\end{itemize}
In particular, if $\KEX_3(\Lambda,\mu)$ would fail
 then $\KEX_2(\lambda,\Mu)$ must fail too.

% Talk about assumption that if KEX_2 works then KEX_3 works?
If these are all read as empty, then our description below reduces
to Taler's existing refresh protocol. 

\smallskip

A coin $(C,\Mu,S)$ consists of 
  a Ed25519 public key $C = c G$,
  a post-quantum public key $\Mu$, and
  an RSA-FDH signature $S = S_d(C || \Mu)$ by a denomination key $d$.
A coin is spent by signing a contract with $C$.  The contract must
specify the recipiant merchant and what portion of the value denoted
by the denomination $d$ they recieve.
If $\Mu$ is large, we may replace it by $H(C || \Mu)$ to make signing
contracts more efficent.

There was of course a blinding factor $b$ used in the creation of
the coin's signature $S$.  In addition, there was a private seed $s$
used to generate $c$, $b$, and $\mu$, but we need not retain $s$
outside the refresh protocol.
$$ c = H(\textrm{"Ed25519"} || s)
\qquad \mu = \CSK(s)
\qquad b = H(\textrm{"Blind"} || s) $$

\smallskip

We begin refresh with a possibly tainted coin $(C,\Mu,S)$ that
we wish to refresh into $n \le \theta$ untainted coins.  

In the change sitaution, our coin $(C,\Mu,S)$ was partially spent and 
retains only a part of the value determined by the denominaton $d$.
There is usually no denomination that matchets this risidual value
so we must refresh from one coin into $n \le \theta$.

For $x$ amongst the symbols $c$, $C$, $\mu$, $\Mu$, $b$, and $s$,
we let $x_{j,i}$ denote the value normally denoted $x$ of
 the $j$th cut of the $i$th new coin being created. 
% So $C_{j,i} = c_{j,i} G$, $\Mu_{j,i}$, $m_{j,i}$, and $b^{j,i}$
%  must be derived from $s^{j,i}$ as above.
We need only consider one such new coin at a time usually, 
so let $x'$ denote $x_{j,i}$ when $i$ and $j$ are clear from context.
In other words, $c'$, $\mu'$, and $b_j$ are derived from $s_j$,
 and both $C' = c' G$ and $\Mu' = \CSK(s')$.

\paragraph{Wallet phase 1.}
\begin{itemize}
\item  For $j=1 \cdots \kappa$:
   \begin{itemize}
   \item  Create random $\zeta_j$ and $l_j$.
   \item  Also compute $L_j = l_j G$.
   \item  Generate $\lambda_j = \LSK((j,i),\mu)$,
          $\Lambda_j = \LPK(\lambda_j)$, and
            $\eta_j = \KEX_2(\lambda_j,\Mu)$.
   \item  Set the linking commitment $\Gamma_{j,0} = (L_j,E_{l_j C}(\Lambda_j))$. 
   \item  Set $k_j = H(l_j C || \eta_j)$.
\smallskip
   \item  For $i=1 \cdots n$:
      \begin{itemize}
      \item  Set $s' = H(\zeta_j || i)$.
      \item  Derive $c'$, $m'$, and $b'$ from $s'$ as above.
      \item  Compute $C' = c' G$ and $\Mu' = \CPK(m')$ too.
      \item  Compute $B_{j,i} = B_{b'}(C' || \Mu')$.
      \item  Encrypt $\Gamma'_{j,i} = E_{k_j}(s')$. 
      \item  Set the coin commitments $\Gamma_{j,i} = (\Gamma'_{j,i},B_{j,i})$
\end{itemize}
\smallskip
\end{itemize}
\item  Send $(C,\Mu,S)$ and the signed commitments
   $\Gamma_* = S_C( \Gamma_{j,i} \quad\textrm{for}\quad j=1\cdots\kappa, i=0 \cdots n )$.
\end{itemize}

\paragraph{Exchange phase 1.}
\begin{itemize}
\item  Verify the signature $S$ by $d$ on $(C || \Mu)$.
\item  Verify the signatures by $C$ on the $\Gamma_{j,i}$ in $\Gamma_*$.
\item  Pick random $\gamma \in \{1 \cdots \kappa\}$.
\item  Mark $C$ as spent by saving $(C,\gamma,\Gamma_*)$.
\item  Send $\gamma$ as $S(C,\gamma)$.
\end{itemize}

\paragraph{Wallet phase 2.}
\begin{itemize}
\item  Save $S(C,\gamma)$.
\item  For $j = 1 \cdots \kappa$ except $\gamma$:
   \begin{itemize}
   \item  Create a proof $\lambda_j^{\textrm{proof}}$ that
          $\lambda_j$ is compatable with $\Lambda_j$ and $\Mu$.
   \item  Set a responce tuple
          $R_j = (\zeta_j,l_j,\lambda_j,\lambda_j^{\textrm{proof}})$.
   \end{itemize}
\item  Send $S_C(R_j \quad\textrm{for}\quad j \ne \gamma )$.
\end{itemize}

\paragraph{Exchange phase 2.}
\begin{itemize}
\item  Verify the signature by $C$.
\item  For $j = 1 \cdots \kappa$ except $\gamma$:
   \begin{itemize}
   \item  Compute $\eta_j = \KEX_2(\lambda_j,\Mu)$.
   \item  Verify that $\Lambda_j = \LPK(\lambda_j)$
   \item  Set $k_j = H(l_j C || \eta_j)$.
   \item  For $i=1 \cdots n$:
     \begin{itemize}
     \item  Decrypt $s' = D_{k_j}(\Gamma'_{j,i})$.
     \item  Compute $c'$, $m'$, and $b'$ from $s_j$.
     \item  Compute $C' = c' G$ too.
     \item  Verify $B' = B_{b'}(C' || \Mu')$.
     \end{itemize}
   \end{itemize}
\item  If verifications all pass then send $S_{d_i}(B_\gamma)$.
\end{itemize}

We could optionally save long-term storage space by
replacing $\Gamma_*$ with both $\Gamma_{\gamma,0}$ and
 $S_C(\Eta_{j,i} \quad\textrm{for}\quad j \ne \gamma )$.
It's clear this requires the wallet send that signature in some phase,
but also the wallet must accept a phase 2 responce to a phase 1 request.

\smallskip

There is good reason to fear tax evasion committed during the
initial withdrawal of a coin as well.  A merchant simply provides
the customer with a blinded but unpurchased coin and asks them to
pay to withdraw it.

\subsection{Withdrawal}\label{subsec:withdrawal}

In Taler, we may address tax fraud on initial withdrawal by turning
withdrawal into a refresh from a pseudo-coin $(C,\Mu)$ in which 
 $C$ is the user's reserve key \cite[??]{Taler} and
 $\Mu$ s a post-quantum public key kept with $C$.
We see below however that our public key algorithm has very different
security requirements in this case, impacting our algorithm choices.


\section{Post-quantum key exchanges}

In \cite{SIDH?,SIDH16}, there is a Diffie-Helman like key exchange
(SIDH) based on computing super-singular eliptic curve isogenies 
which functions as a drop in replacement, or more likely addition,
for Taler's refresh protocol.

In SIDH, private keys are the kernel of an isogeny in the 2-torsion
or the 3-torsion of the base curve.  Isogenies based on 2-torsion can
only be paired with isogenies based on 3-torsion, and visa versa.  
This rigidity makes constructing signature schemes with SIDH hard
\cite{}, but does not impact our use case.  

We let $\mu$ and $\Mu$ be the SIDH 2-torsion private and public keys,
repectively.  We simlarly let $\lambda$ and $\Lambda$ be the
SIDH 3-torsion private and public keys.  

We envision the 2-torsion secret key generation function $\CSK(s)$
for $\mu$ being deterministic with seed $s$, but the 3-torsion secret
key generation function $\LSK()$ ignores the arguments given above 
Our 2-torsion and 3-torsion public key derivation functions
$\CPK(\mu)$ and $\LPK(\lambda)$ along with our two key derivation
functions $\KEX_2$ and $\KEX_3$, all work as described in
\cite[\S6]{SIDH16}.

% We refer to \cite[\S6]{SIDH16}, \cite[??]{SIDH?15?}, and
% \cite[]{SIDH?11?} for further discussion of the implementation.

There is no relationship between $\mu$ and $\lambda$ in SIDH, so
$\Lambda$ cannot itself leak any information about $C$.

\smallskip

Ring-LWE based key exchanges like \cite{Peikert14,NewHope} require
that both Alice and Bob's keys be ephemeral because the success or
failure of the key exchange leaks one bit about both keys\cite{}.
As a result, authentication with Ring-LWE based schemes remains
harder than with discrete log schemes\cite{}.

We observe however that the Taler wallet controls both sides during 
the refresh protocol, so the wallet can ensure that the key exchange 
always succeeds.  In fact, the Ring-LWE paramaters could be tunned to
leave the probability of failure rather high, saving the exchange
bandwidth, storage, and verification time.  

We let $\mu$ and $\Mu$ be Alice (initator) side the private and public
keys, repectively.  We simlarly let $\lambda_j$ and $\Lambda_j$ be the
Bob (respondent) private and public keys. 
% DO IT :
Again now, $\CPK$, $\CSK$, $\LPK$, $\LSK$, $\KEX_2$ and $\KEX_3$
can be defined from \cite{Peikert14,NewHope}.  

\smallskip

At present, the SIDH implemention in \cite{SIDH16} requires about
one third the key material and 100?? times as much CPU time as the
Ring-LWE implemention in \cite{NewHope}.
As noted above, Ring-LWE admits significant optimizations for the
Taler refresh situation.

We observe however that the Ring-LWE public key $\Lambda$ has a risk
of leaking information about $\Mu$.  In particular, if polynomial $a$
depends upon $\Mu$, like in \cite{NewHope}, then anonymity explicity
depends upon the Ring-LWE problem\cite{}.  
...


\section{Hashed-based one-sided public keys}

We now define our hash-based encryption scheme.
Let $\delta$ denote our query security paramater and
 let $\mu$ be a bit string.
For $j \le \kappa$, we define a Merkle tree $T_j$ of height $\delta$
with leaves $\eta_j = H(\mu || "YeyCoins!" || t || j)$
 for some leaf index $t \le 2^\delta$. 
Let $t_j$ denote the root of $T_j$.
Set $\Mu = H(t_1 || \cdots || t_\kappa)$,
 which defines $\CPK(\mu)$.

Now let $\lambda_j = \LSK((t,j),\mu)$ consist of
$(t,j,\eta_j)$ along with both
 the Merkle tree path that proves $\eta_j$ is a leaf of $T_j$,
and $(t_1,\ldots,t_\kappa)$,
 making $\LSK$ an embelished Merkle tree path function.
Also let $\Lambda_j = \LPK(\lambda_j)$ be $(t,j)$

We define $\KEX_2(\lambda_j,\Mu)$ to be $\eta_j$
 if $\lambda_j$ proves that $\eta_j$ is the $t$ leaf for $\Mu$,
or empty otherwise.
$\KEX_3(\Lambda_j,\mu)$ simply recomputes $\eta_j$ as above.
If $\KEX_2$ works then so does $\KEX_3$.

As $\Lambda_j = (t,j)$, it matters that $\lambda_j$ actually
demonstrates the position $t$ in the Merkle tree.

... proofs ...

\smallskip

We observe that $\CPK$ has running time $O(2^\delta)$, severely
limiting $\delta$.  We lack the time-space trade offs resolve
this issue for hash-based signature (see \cite{SPHINCS}).

Imagine that $\delta = 0$ so that $T_j = H(\eta_j)$.
In this scenario, a hostile exchange could request two different
$\gamma$ to learn all $\eta_j$, if the wallet reruns its second
phase.  In principle, the wallet saves the exchange's signed 
choice of $\gamma$ before revealing $\eta_j$ for $j \neq \gamma$.
It follows that even $\delta = 0$ does technically provides
post-quantum anonymity,

We must worry about attacks that rewind the wallet from phase 2 to
phase 1, or even parallelism bugs where the wallet answer concurrent
requests.  We cannot remove the curve25519 exchange $l_j C$ from the
refresh protocol becausenn such attacks would represent serious risks
without it.  With our $l_j C$ component, there is little reason for
an attacker to pursue $\eta_j$ alone unless they expect to break
curve25519 in the future, either through mathematical advances or
by building a quantum computer.  

We therefore view $\delta$ as a query complexity paramater whose 
optimial setting depends upo nthe strength of the overall protocoll. 

\smallskip

We can magnify the effective $\delta$ by using multiple $\eta_j$.

... analysis ...
% multiple withdrawals

We believe this provides sufficent post-quantum security for
refreshing change.  


\section{Hash and Ring-LWE hybrid}

We noted in \S\ref{subsec:withdrawal} above that exchange might
require that initial withdrawals employs a refresh-like operation.
In this scenarion, we refresh from a pseudo-coin $(C,\Mu)$ where
 $C$ is the user's reserve key \cite[??]{Taler} and
 $\Mu$ s a post-quantum public key kept with $C$.
As a result, our hash-based scheme should increase the security
paramater $\delta$ to allow a query for every  withdrawal operation.

Instead, we propose using a Merkle tree of Alice side Ring-LWE keys,
while continuing to invent the Bob side Ring-LWE key.  

...
% Use birthday about on Alice vs Bob keys

\section{Conclusions}


\bibliographystyle{alpha}
\bibliography{taler,rfc}

% \newpage
% \appendix

% \section{}



\end{document}

\begin{itemize}
\item 
\item 
\end{itemize}






Crazy pants ideas : 

Use a larger Mrkle tree with start points seeded throughout 

Use a Merkle tree of SWIFFT hash functions becuase
 their additive homomorphic property lets you keep the form of a polynomial



